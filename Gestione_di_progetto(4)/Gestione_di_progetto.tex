\input{../Templates/layout}
\input{../Templates/commands}
\begin{document}
\section{Gestione di Progetto(4)}

Fondamenti della \textbf{Gestione di progetto:}
\begin{enumerate}
	\item Introdurre processi nel progetto (introdurre disciplina);
	\item Stimare i costi e le risorse necessarie (avere una pianificazione preventiva) infatti un progetto diventa fattibile se la stima e accettabile;
	\item Pianificazione e assegnazione delle attività;
	\item Controllare le attività e verificare i risultati (avere chiaro e presente lo stato di avanzamento), non devo fare un pull \textit{chiedere} deve essere già tutto pronto. L'uso di \textbf{Milestone} aiuta a capire la differenza tra attesa e realtà.
\end{enumerate}

Per molto tempo si è pensato che i sw fossero repliche uniche ("\textit{one off}"); grave errore creare cose che siano irripetibili, la nostra attività prevalente sarà la \textit{manutenzione}.

Vediamo alcuni \textbf{Fattori di rischio} per la gestione di progetto:

\begin{itemize}

	\item \textbf{Variabilità nel personale:} nella disponibilità e nella composizione del team. Le persone possono "\textit{sparire}", a quel punto bisogna trovare un rimpiazzo;
	
	\item \textbf{Tecnologia:} due tecniche: chi usa solo tecnologie consolidate, e chi usa tecnologia innovativa (per motivi di competizione) ma fortemente instabile. Questa variabilità è un grande rischio.
	
	\item \textbf{Mercato:} competizione sul mercato.
	\item \textbf{Requisiti:} possono cambiare;
	\item \textbf{Specifiche:} ritardo nella presentazione;

\end{itemize}
Gli ultimi due sono facilmente evitabili con un'adeguata gestione dei rischi che va pianificata sia all'inizio che in corso d'opera in quanto possono variare nel tempo (analisi di probabilità che emergano). Ecco come pianificare una buona gestione dei rischi:

\begin{enumerate}

	\item \textbf{Identificazione:} capire quali sono nelle varie categorie specificate sopra;
	
	\item \textbf{Analisi:} probabilità di occorrenza;
	
	\item \textbf{Pianificazione:} come evitare i rischi;
	
	\item \textbf{Controllo:} attenzione continua tramite rilevazione di indicatori.

\end{enumerate}

Il team necessita di \textbf{ruoli}, che identificano capacità e compiti. Un ruolo è la \textit{funzione aziendale} assegnata al progetto. Vediamo alcuni ruoli:

\begin{itemize}

	\item \textbf{Qualità:} ciò che ci consente di puntare al miglioramento continuo di efficienza ed efficacia. Qualità di prodotto è diverso da qualità di processo. E' più importante la qualità di processo;
	
	\item \textbf{Sviluppo:} insieme delle competenze per la parte di attività tecnica e realizzativa del prodotto;
	
	\item \textbf{Direzione:} fa sì che l'organizzazione possa stare in piedi;
	
	\item \textbf{Amministrazione:} ("\textit{service manager}") erogano/gestiscono l'infrastruttura che aiuta a fare il proprio lavoro (es. manutenzione e sicurezza). Se esiste una buona infrastruttura si lavora meglio.

\end{itemize}

Competenze allo richieste:

\begin{itemize}

	\item \textbf{Analista:} colui che serve per fare analisi dei requisiti; aiuta l'avanzamento di maturità dei requisiti; bisogna avere competenze su più fronti, sapere ascoltare gli \textit{stakeholder}, scrivere requisiti \textit{bounded} e \textit{coherent}, ragionevoli, utili, realizzabili e verificabili. Hanno un gran peso sul successo del progetto; l'analista pensa al problema, non alla soluzione;
	
	\item \textbf{Progettista:} pensa alla soluzione. Deve avere competenze tecnologiche e tecniche. Ricevuto il problema elabora la migliore soluzione possibile nel rispetto dei vincoli finanziari e temporali. Deve sapere di architettura, mettere insieme le parti della soluzione (divide et impera), rendere il problema piccolo affinché sia dominabile. Un solo progettista in un team;
	
	\item \textbf{Programmatori:} hanno responsabilità e visione circoscritte, il passaggio di consegna tra progettista e programmatore deve essere chiaro. Partecipano anche alla manutenzione. Ho tanti programmatori quanti posso averne;
	
	\item \textbf{Verificatori:} la validazione si fa sul prodotto finito. Impiega almeno 1/3 del tempo. Partecipano a tutto il ciclo di vita del progetto. Include anche la verifica del codice: 1) impone al programmatore stesso di verificare il suo codice, 2) lo fa verificare ad una terza persona indipendente;
	
	\item \textbf{Responsabile:} ce ne sarà 1. Rappresenta il progetto presso il fornitore e presso il committente. E' quello che fa da intermediario, che dice "\textit{come siamo messi}"; pianifica, gestisce risorse e rischi, coordina. Responsabilità su relazioni esterne. Deve sapere ciò di cui parla. Diventa responsabile dopo avere acquisito esperienza su altri ruoli;
	
	\item \textbf{Amministratore:} ruolo molto utile, prepara l'ambiente di lavoro, strumenti di collaborazione e controllo di avanzamento. Gestione della documentazione di progetto. Risoluzione di problemi legati alla gestione di processi. Assegnazione di \textit{tickets}, un incarico con scadenza assegnato ed accettato.


\end{itemize}

\end{document}
