%%%%%%%%%%%%%%%%%%%%%%%%%%%%%%%%%%%%%%%%%
% Structured General Purpose Assignment
% LaTeX Template
%
% This template has been downloaded from:
% http://www.latextemplates.com
%
% Original author:
% Ted Pavlic (http://www.tedpavlic.com)
%
% Note:
% The \lipsum[#] commands throughout this template generate dummy text
% to fill the template out. These commands should all be removed when 
% writing assignment content.
%
%%%%%%%%%%%%%%%%%%%%%%%%%%%%%%%%%%%%%%%%%

\documentclass{article}

\usepackage{fancyhdr} % Required for custom headers
\usepackage{lastpage} % Required to determine the last page for the footer
\usepackage{extramarks} % Required for headers and footers
\usepackage{graphicx} % Required to insert images
\usepackage[utf8]{inputenc}

% Margins
\topmargin=-0.45in
\evensidemargin=0in
\oddsidemargin=0in
\textwidth=6.5in
\textheight=9.0in
\headsep=0.25in 

\linespread{1.1} % Line spacing



\setlength\parindent{0pt} % Removes all indentation from paragraphs

%----------------------------------------------------------------------------------------
%	DOCUMENT STRUCTURE COMMANDS
%	Skip this unless you know what you're doing
%----------------------------------------------------------------------------------------

% Header and footer for when a page split occurs within a problem environment
\newcommand{\enterProblemHeader}[1]{
\nobreak\extramarks{#1}{#1 continued on next page\ldots}\nobreak
\nobreak\extramarks{#1 (continued)}{#1 continued on next page\ldots}\nobreak
}

% Header and footer for when a page split occurs between problem environments
\newcommand{\exitProblemHeader}[1]{
\nobreak\extramarks{#1 (continued)}{#1 continued on next page\ldots}\nobreak
\nobreak\extramarks{#1}{}\nobreak
}

\setcounter{secnumdepth}{0} % Removes default section numbers
\newcounter{homeworkProblemCounter} % Creates a counter to keep track of the number of problems

%----------------------------------------------------------------------------------------
%	NAME AND CLASS SECTION
%----------------------------------------------------------------------------------------

\newcommand{\lessonNumber}[1]{Lezione\ \##1} % Assignment title
\newcommand{\lessonDate}[4]{#1,\ #2\ #3\ #4} % Due date
\newcommand{\lessonCourse}[1]{#1} % Course/class
\newcommand{\lessonTime}[1]{#1} % Class/lecture time
\newcommand{\lessonTeacher}[1]{#1} % Teacher/lecturer
\newcommand{\lessonAuthor}[1]{#1} % Your name
\begin{document}

\section{Qualità del software(6)}
Il concetto di qualità si lega strettamente a quello di valutazione, la qualità ha più aspetti e aspettative, la loro soddisfazione ha più destinatari: \textit{chi fa, chi usa, chi valuta}.

\fbox{\textbf{Def:}Insieme delle caratteristiche di un'entità (prodotto, processo, servizio) che ne determinano la capacità di soddisfare esigenze espresse e implicite} 
E' importante vedere che le esigenze possono essere \textbf{espresse} o \textbf{implicite}. Spesso la parte implicita è dominante e bisogna scoprirla. La qualità può essere guardata da 3 diversi punti di vista:

\begin{itemize}

	\item \textbf{Visione relativa e comparativa:} in relazione alle altre alternative disponibili;
	\item \textbf{Intrinseca:} hai qualità se soddisfi i bisogni, è chiaro che deve essere così, dimensione ovvia e non comparativa;
	\item \textbf{Quantitativa:} anche se non ho competitività oggi, un domani potrò confrontarmi. Mi dà una posizione definitiva (es. stelle degli hotel) anche senza competizione.

\end{itemize} 

La qualità viene erogata con un processo che si chiama \textbf{gestione di qualità}\fbox{\textbf{Def:} La struttura organizzativa, le responsabilità, le procedure, i procedimenti e le risorse messe in atto per il perseguimento della qualità.}

Chi lavora nei principi del SWE dovrebbe essere conforme a queste caratteristiche. Ambito del sistema di qualità.

\begin{itemize}

	\item \textbf{Pianificazione:} voglio che le attività siano sistematiche nel modo richiesto; la pianificazione è alla base di ogni sistema di qualità. Senza pianificazione è molto più facile fallire;
	\item \textbf{Controllo:} ogni attività svolta può introdurre errori e devo cercare di intercettarli; un sistema di qualità ha bisogno di verifica;
	\item \textbf{Miglioramento continuo}.

\end{itemize}
%slide 5 qualità del software

\end{document}
