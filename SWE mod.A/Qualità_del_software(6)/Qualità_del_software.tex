\input{../Templates/layout}
\input{../Templates/commands}
\begin{document}

\section{Qualità del software(6)}
Il concetto di qualità si lega strettamente a quello di valutazione, la qualità ha più aspetti e aspettative, la loro soddisfazione ha più destinatari: \textit{chi fa, chi usa, chi valuta}.

\textbf{Qualità:} \texttt{Insieme delle caratteristiche di un'entità (prodotto, processo, servizio) che ne determinano la capacità di soddisfare esigenze espresse e implicite.}\\
E' importante vedere che le esigenze possono essere \textbf{espresse} o \textbf{implicite}. Spesso la parte implicita è dominante e bisogna scoprirla. La qualità può essere guardata da 3 diversi punti di vista:

\begin{itemize}

	\item \textbf{Visione relativa e comparativa:} in relazione alle altre alternative disponibili;
	\item \textbf{Intrinseca:} hai qualità se soddisfi i bisogni, è chiaro che deve essere così, dimensione ovvia e non comparativa;
	\item \textbf{Quantitativa:} anche se non ho competitività oggi, un domani potrò confrontarmi. Mi dà una posizione definitiva (es. stelle degli hotel) anche senza competizione.

\end{itemize} 

La qualità viene erogata con un processo che si chiama \textbf{gestione di qualità}: \texttt{La struttura organizzativa, le responsabilità, le procedure, i procedimenti e le risorse messe in atto per il perseguimento della qualità.}

Chi lavora nei principi del SWE dovrebbe essere conforme a queste caratteristiche. Ambito del sistema di qualità.

\begin{itemize}

	\item \textbf{Pianificazione di qualità:} voglio che le attività siano sistematiche nel modo richiesto; la pianificazione è alla base di ogni sistema di qualità. Senza pianificazione è molto più facile fallire. \\
	\texttt{Attività del sistema qualità mirate a fissare gli obbiettivi di qualità. i processi e le risorse necessarie per conseguirli;}
	\item \textbf{Controllo di qualità:} ogni attività svolta può introdurre errori e devo cercare di intercettarli; un sistema di qualità ha bisogno di verifica.\texttt{ Le attività del sistema qualità pianificate e attuate affinché il prodotto soddisfi i requisiti attesi}, per esempio la \textit{quality assurance} deve essere \textbf{preventiva} invece che correttiva, dobbiamo diventare proattivi;
\end{itemize}
Gli standard ci aiutano a migliorare la qualità in quanto sono una raccolta organica di best practice, e come vantaggio portano anche un elemento di continuità per i nuovi assunti. D'altro canto però questi possono essere visti come bloccanti e in effetti la loro attuazione cieca può portare eccessi di burocrazia.
L'attenzione alla qualità deve spostarsi dal prodotto al \textbf{sistema} e alla sua organizzazione. La qualità di prodotto è meno importante della qualità di sistema.\\
Il sistema è un insieme di attività organizzate e coese.\\
Nel caso del \textbf{Ciclo di Deming} pianifico attività che producono miglioramenti. Tutti i processi hanno come prima attività l'istanziazione del processo e poi la pianificazione. Per poter migliorare devo prima misurare, e controllare il miglioramento avvenuto.
Il \textit{Ciclo di Deming} è formato da 4 punti:
	\begin{itemize}
		\item \textbf{Plan:} pianifico gli obbiettivi che il processo in analisi deve raggiungere, cosa deve essere realizzato e come andrà controllato.
		\item \textbf{Do:} svolgo il processo secondo la pianificazione.
		\item \textbf{Ceck:} controllo dove sono arrivato rispetto agli obbiettivi prefissati, in questo modo posso intervenire in tempo se ci fossero problemi o ritardi e posso inoltre andare a migliorare i risultati.
		\item \textbf{Act:} attuo le correzioni pianificate nel punto precedente.
	\end{itemize}

\textbf{Modello di qualità:} rappresentazione astratta, insieme di strumenti che servono a valutare la qualità.\\ 
\textbf{Modello di Bohem:} la qualità viene descritta da un insieme di caratteristiche fissate e non arbitrarie. Bohem ne ha definite 7 e le ha suddivise in ulteriori 15 sotto categorie. Secondo ISO/IEC 9126:2001, che è un'evoluzione di Bohem, ho i seguenti principi:

\begin{enumerate}

	\item \textbf{Funzionalità:} avere le funzionalità attive è qualità;
	\item \textbf{Affidabilità};
	\item \textbf{Efficienza:} devo metterci poco tempo, quante risorse uso per fare una determinata cosa;
	\item \textbf{Usabilità:} non vanno bene cose troppo complesse per gli utilizzatori;
	\item \textbf{Manutenibilità};
	\item \textbf{Portabilità}.
	\item \textbf{Qualità in uso} è una dimensione specifica rivolta esclusivamente all'utente finale.
\end{enumerate}

\textbf{Software metrics:} qualsiasi tipo di misura che riguarda un sistema software, un processo o la documentazione. Questo fa si che abbia prodotti e processi valutabili

\end{document}
