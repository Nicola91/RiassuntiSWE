\input{../Templates/layout}
\input{../Templates/commands}
\begin{document}

\section{Qualità del software(6)}
Il concetto di qualità si lega strettamente a quello di valutazione, la qualità ha più aspetti e aspettative, la loro soddisfazione ha più destinatari: \textit{chi fa, chi usa, chi valuta}.

\fbox{\textbf{Def:}Insieme delle caratteristiche di un'entità (prodotto, processo, servizio) che ne determinano la capacità di soddisfare esigenze espresse e implicite} 
E' importante vedere che le esigenze possono essere \textbf{espresse} o \textbf{implicite}. Spesso la parte implicita è dominante e bisogna scoprirla. La qualità può essere guardata da 3 diversi punti di vista:

\begin{itemize}

	\item \textbf{Visione relativa e comparativa:} in relazione alle altre alternative disponibili;
	\item \textbf{Intrinseca:} hai qualità se soddisfi i bisogni, è chiaro che deve essere così, dimensione ovvia e non comparativa;
	\item \textbf{Quantitativa:} anche se non ho competitività oggi, un domani potrò confrontarmi. Mi dà una posizione definitiva (es. stelle degli hotel) anche senza competizione.

\end{itemize} 

La qualità viene erogata con un processo che si chiama \textbf{gestione di qualità}\fbox{\textbf{Def:} La struttura organizzativa, le responsabilità, le procedure, i procedimenti e le risorse messe in atto per il perseguimento della qualità.}

Chi lavora nei principi del SWE dovrebbe essere conforme a queste caratteristiche. Ambito del sistema di qualità.

\begin{itemize}

	\item \textbf{Pianificazione:} voglio che le attività siano sistematiche nel modo richiesto; la pianificazione è alla base di ogni sistema di qualità. Senza pianificazione è molto più facile fallire;
	\item \textbf{Controllo:} ogni attività svolta può introdurre errori e devo cercare di intercettarli; un sistema di qualità ha bisogno di verifica;
	\item \textbf{Miglioramento continuo}.

\end{itemize}
%slide 5 qualità del software

\end{document}
