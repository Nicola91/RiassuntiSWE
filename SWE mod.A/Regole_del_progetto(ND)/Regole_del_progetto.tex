\input{../Templates/layout}
\input{../Templates/commands}
\begin{document}

\section{Regole del Progetto (ND)}

C'è un rapporto strutturato fra i gruppi di progetto (fornitori) e il docente che rappresenta i proponenti (committente). Il cliente fissa un calendario e a una certa data vuole vedere determinate cose (Revisione di avanzamento). A questa revisione si accede facendo delle operazioni obbligatorie e vincolanti. I documenti rappresentano l'oggetto di discussione sull'avanzamento.

\includegraphics[width=0.75\columnwidth]{img1}
%arrivati fin qua
Si esce dalla RR se i requisiti stanno tra \textbf{bounded} (perimetrati) e \textbf{acceptable} (negoziati). Arrivare ad avere la base di lavoro. Usciti da RP i requisiti sono tra acceptable e \textbf{addressed} e devo avere una buona architettura che soddisfi i requisiti (\textbf{architecture selected}. Uscito da RQ sono tra addressed e \textbf{fullfilled}, la progettazione e la codifica sono \textbf{usable}. Quando esco dalla RA ho i requisiti fullfilled e la progettazione è pronta (\textbf{ready}) per andare sul mercato.\\\\

Temporalmente c'è almeno un mese tra una revisione e l'altra. Le revisioni hanno negli strati del SWE due possibili accezioni:

\begin{itemize}

	\item \textbf{Revisione formale}, c'è un arbitro esterno che dà un giudizio;
	\item \textbf{Revisioni di progresso}, sono delle discussioni paritarie, è come se ci fosse un consulente che valuta.

\end{itemize}

Le revisioni formali sono la prima (RR) e l'ultima (RA). Le altre sono di progresso, che servono per aiutare il gruppo a misurare il proprio avanzamento. Su ogni revisione ci sarà un voto (che fa media) e quelle formali sono \textbf{bloccanti}. Il voto di progetto conta per il 60\% del voto totale. Le revisioni si svolgono in ingresso con la presentazione di documentazione e poi con un colloquio orale e una discussione. Nelle revisioni interne ci sono degli obiettivi \textbf{tecnici} e \textbf{gestionali}. Per gestionali si intende l'avere una evidenza oggettiva che il progetto è sotto controllo per tempi, costi e avanzamento. Mostrare che l'avanzamento è coerente con le attese rispetto al modello di sviluppo utilizzato. Bisogna avere una \textbf{baseline}. Quando essa esiste allora siamo in grado di dimostrare che il progetto ha un progresso.\\\\

Nella Revisione dei Requisiti si entra con 3 documenti esterni e uno interno:

\begin{itemize}

	\item \textbf{Analisi dei requisiti}, spiega i requisiti derivati dal capitolato;
	\item \textbf{Piano di qualifica}, mette insieme verifica e validazione, che costituiscono la base della \textbf{qualità}, ovvero tutto ciò che facciamo per ottenerla;
	\item \textbf{Piano di progetto}, la strategia per l'uso delle risorse intese come tempi, persone e gestione di queste due;

\end{itemize}

Il documento interno sono le \textbf{Norme di progetto}.\\
Nella Revisione di Progettazione devo avere due cose:

\begin{itemize}

	\item \textbf{Progettazione di alto livello};
	\item \textbf{Progettazione di dettaglio}.

\end{itemize}

Devo portare un documento che si chiama \textbf{specifica tecnica} e l'avanzamento incrementale di PQ e del PP in cui avremo un pezzo di consuntivo. Porto la \textbf{Definizione di prodotto}, dalla quale il programmatore inizierà il suo lavoro. Devo informare il cliente sulle caratteristiche del prodotto realizzato.\\
In RQ è finita la codifica oppure mancano pezzi non essenziali. Approvazione dell'esito finale della fase di verifica. Dovrò aggiornare la PQ e il PP e presentare una bozza del manuale utente.\\
Nella RA porto il prodotto finale \textbf{ultimato}. Adempimento del fornitore:

\begin{center}
\includegraphics[width=0.75\columnwidth]{img5} % Example image
\end{center}
\end{document}