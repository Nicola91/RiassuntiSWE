\input{../Templates/layout}
\input{../Templates/commands}
\begin{document}

\section{Qualit� di processo(7)}

\textit{''Dai tubi sporchi non esce acqua pulita''}\\
Voglio tecniche produttive. Quanto pi� sono produttivo tanto meno sar� reattivo. La \textbf{quality assurance} � fatta da regole, procedure e strumenti. La verifica la faccio a valle di un lavoro svolto che mi � gi� costato. Al tempo 0 ho costo 0. Voglio imparare ad essere massimamente proattivo per risparmiare risorse e fare meno fatica. E gli assi di risparmio sono assi molto importanti e rilevanti. L'intelligenza del processo sta nella sua capacit� di migliorarsi, e per migliorarsi ci si valuta, cio� ci si misura rispetto a obiettivi.

\begin{center}
\includegraphics[width=0.75\columnwidth]{img1.png} % Example image
\end{center}

Dobbiamo crearci regole per il miglioramento e per farlo dobbiamo fare il lavoro dei processi e dei controlli. Non posso mettere il controllo su un processo non definito, quindi il primo passo � definire il processo. Ci concentreremo sulle caratteristiche che se sbagliamo ci costeranno di pi� (esempio cose che dovranno essere rifatte o che avranno conseguenze sul prodotto finale). \textbf{Norme ISO 9000-1}, � importante perch� vale come garanzia per il cliente, certifica che si lavora ad un certo livello.

\begin{center}
\includegraphics[width=0.75\columnwidth]{img2.png} % Example image
\end{center}


Politica di qualit� da fissare, una volta fissata posso fare un \textbf{manuale della qualit�} da cui deriva un \textbf{piano di qualit�} specifico del progetto. Le attuazioni operative del piano di qualit� includono alcune operazioni molto semplici.\\\\
\textbf{CMM}, strumento che serve per valutare come lavoriamo (\textit{Capability Maturity Model}).

\end{document}