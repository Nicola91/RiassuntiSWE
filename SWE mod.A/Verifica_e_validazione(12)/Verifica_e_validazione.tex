\input{../Templates/layout}
\input{../Templates/commands}
\begin{document}

\section{Verifica e validazione(12)}


\begin{itemize}

	\item \textbf{Verifica:} "ho fatto il sistema nel modo giusto", accerta che l'esecuzione delle attività di processi svolti nella fase in esame non abbia introdotto errori nel prodotto;
	\item \textbf{Validazione:} "ho fatto il sistema giusto", accerta che il prodotto realizzato sia conforme alle attese.

\end{itemize}

La \textit{software verification} ricerca la completezza e la correttezza del software e tratta ciò che lo supporta. Consente di valutare di conseguenza che il sw sia validato. La verifica è a supporto della validazione e la validazione è l'ultima cosa che faccio in un progetto. La verifica è un attività che svolgo durante \textbf{tutto lo sviluppo} fino all'ultimo istante dove farò validazione, che servirà a dire che ciò che ho fatto è la cosa giusta. La verifica va fatta per impedire che la risposta finale non sia sbagliata. Devo garantire tre cose importanti:

\begin{itemize}

	\item \textbf{Consistenza:} "sono ciò che vi attendevate fossi";
	\item \textbf{Correttezza:} "ciò che ho conseguito è corretto rispetto alle norme";
	\item \textbf{Completezza:} "tutto ciò che ho creato è tutto ciò era atteso".

\end{itemize}

Sono tre caratteristiche di cui devo accertare l'esistenza su tutti i prodotti parziali dello sviluppo. Il verificatore impara le norme e dice che quello che è stato fornito è fatto come richiesto. Non corregge nè rifà il lavoro ma controlla solo che tutto rispetti le tre caratteristiche. La validazione conseguentemente è una conferma \textbf{by examination}, mostra copertura dei requisiti (utente e sw).

Per il verificatore ho due forme di \textbf{analisi}:

\begin{itemize}

	\item \textbf{Analisi statica:} non richiede l'esecuzione del programma, studia le caratteristiche del codice sorgente (e a volte anche del codice oggetto), conformità a regole date, assenza di difetti, presenza di proprietà positive;
	\item \textbf{Analisi dinamica:} richiede l'esecuzione del programma, viene effettuata tramite \textbf{test}, usata sia nella verifica che nella validazione.						\begin{itemize}

			\item \textbf{Ripetibilità:} è un requisito essenziale. Dobbiamo assumere uno stato iniziale prima dell'esecuzione in quanto ha influenza sia diretta che indiretta sull'esecuzione. Il test deve essere \textbf{deterministico} ed eseguire le cose secondo un ordine noto. \textbf{Specifica di un test};
			\item \textbf{Strumenti}:
				\begin{itemize}

					\item \textbf{Driver:} componente attiva fittizia per pilotare una parte;
					\item \textbf{Stub:} componente passiva fittizia per simulare una parte;
					\item \textbf{Logger:} componente non intrusiva di registrazione dei dati di esecuzione per l'analisi dei risultati. Ogni tanto deve lasciare traccia del suo esito;	
	
				\end{itemize}
	\		item \textbf{Unità:} può essere anche un aggregato di procedure. La più piccola unità sw che è conveniente verificare singolarmente. Un \textit{modulo} è parte dell'unità, un \textit{componente} integra più unità.

		\end{itemize}

\end{itemize}

\includegraphics[width=0.5\columnwidth]{img3} % Example image

Con \textbf{stub} ho dei test \textit{Top down} dalla radice alle foglie, con \textbf{driver} ho dei test \textit{Bottom up} dalle foglie alla radice

Tipi di test

\includegraphics[width=0.5\columnwidth]{img4} % Example image

\begin{itemize}
	\item \textbf{Test di unità:} si svolgono con il massimo grado di parallelismo, la responsabilità è dello stesso programmatore sulle unità più piccole. L'obbiettivo è quello di verificare la correttezza del codice.
	\item \textbf{Test di integrazione:} le componenti vengono verificate e sviluppate in parallelo, rileva errori residui nella realizzazione dei componenti, cambiamenti nelle interfacce, requisiti, integrazione con altre applicazioni non ben conosciute.
	\item \textbf{Test di sistema e collaudo:} dai requisiti so dire quanti test di sistema avrò. I test di sistema è un'attività interna del fornitore per accertare la copertura dei requisiti, il collaudo invece viene supervisionato dal committente.
	\item \textbf{Test di regressione:} è l'insieme dei test che accertano che la modifica di una parte P non causi problemi in P o in altre parte che dipendono da essa, infatti modifiche aggiunte o rimozioni non devono pregiudicare le funzionalità già verificate.
\end{itemize}

\subsection{Analisi Statica}
Si può applicare ai metodi di lettura che si possono suddividere in due tipo:
\begin{itemize}
	\item \textbf{Walkthrought:} l'obbiettivo è quello di rilevare la presenza di difetti, si esegue una lettura a largo spetto senza l'assunzione di presupposti, le fasi sono: pianificazione, lettura, discussione, correzione dei difetti. 
	\item \textbf{Inspection:} l'obbiettivo è sempre quello di rilevare difetti ma eseguendo una lettura mirata, si focalizza la ricerca su presupposti, le fasi sono: pianificazione, definizione della lista di controllo, lettura, correzione dei difetti.
\end{itemize}

\textit{Inspection} è basato su errori presupposti ed è più rapido, \textit{Walkthrought} richiede maggiore attenzione ma è più collaborativo.

Valori dell'\textbf{Analisi Statica}:

\begin{itemize}
	\item \textbf{Funzionalità:} analisi statica come attività preliminare, liste di controllo rispetto ai relativi requisiti \textit{(tutte e solo le funzionalità per tutti e solo i componenti necessari, compatibilità tra tutte le soluzioni adottate)}, valutazione di accuratezza;
	\item \textbf{Affidabilità:} dimostrabile tramite combinazione di prove, analisi statica come attività preliminare, liste di controllo rispetto ai relativi requisiti \textit{(robustezza, capacità di ripristino e recupero da errori, adesione alle norme)}, valutazione di maturità.
	\item \textbf{Usabilità:} le prove sono imprescindibili, analisi statica come attività complementare, liste di controllo rispetto ai manuali d'uso \textit{(comprensibilità, apprendibilità, adesione a norme e prescrizioni)}, questionari sottomessi agli utenti.
	\item \textbf{Efficenza:} le prove sono necessarie, analisi statica come attività complementare, liste di controllo rispetto alle norme di codifica, margini di miglioramento e confidenza grazie alla confidenza acquisita.
	\item \textbf{Manutenibità:} analisi statica come strumento ideale, liste di controllo rispetto a specifiche norme di codifica, e alla prove per accertarne, prove di stabilità.
	\item \textbf{Portabilità:} analisi statica come strumento ideale, liste di controllo rispetto a specifiche norme di codifica, prove come strumento complementare.
\end{itemize}


\end{document}