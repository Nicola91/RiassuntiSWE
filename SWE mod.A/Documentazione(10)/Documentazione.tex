\input{../Templates/layout}
\input{../Templates/commands}
\begin{document}

\section{Documentazione (10)}

La documentazione deve essere una conseguenza dei processi organizzativi. Serve documentare nel modo meno intrusivo possibile. Fra le cose più importanti che fa la documentazione è fornire una misura sull'avanzamento del progetto (Piano di progetto), che fornisce dei \textbf{consuntivi} (stato fotografato) e \textbf{preventivi} (stime). I consuntivi sono finali se siamo alla fine del progetto, o parziali se siamo in corso. Documentare serve per dominare la complessità dei processi produttivi, attenuare la volatilità dei requisiti, facilitare il controllo di avanzamento.

Misureremo le cose sulle quali possiamo e vogliamo fissare degli obiettivi di miglioramento. Misurazione per obiettivi, "ad hoc". Il responsabile di progetto deve avere un ``\textit{cruscotto}'' con gli indicatori delle metriche che utilizziamo, e questi indicatori devono essere aggiornati. Vogliamo inoltre che il responsabile spenda il meno possibile per far questo devo documentare tutte le attività di pianificazione, gestione, sviluppo, verifica e validazione. Il piano ci dà degli obiettivi sui tempi e costi, le norme sono invece gli strumenti e le procedure che uso per rendere il piano fattibile. Il primo e più importante documento da realizzare (interno) sono le \textbf{norme di progetto}, che cresceranno nel tempo. Un altro documento importante è la specifica software, che è la descrizione ad alto livello del sistema.\\
Ogni architettura sw ha molte viste:

\begin{itemize}

	\item \textbf{Modello statico:} Identifica le componenti principali e procede per decomposizione grafica;
	\item \textbf{Modello dinamico:} Illustra la struttura a processi del sistema;
	\item \textbf{Modello delle interfacce:} Definisce le interfacce  del sistema;
	\item \textbf{Modello delle relazioni:} Identifica il flusso dei dati tra componenti distinti in relazione tra loro;
	\item \textbf{Modello di distribuzione:} Associazione tra nodi fisici e componenti logiche.

\end{itemize}

L'architettura va descritta in due documenti distinti:

\begin{itemize}

	\item \textbf{Specifica tecnica};
	\item \textbf{Definizione di prodotto:} ho bisogno di una \textit{milestone} che dica che questa è una buona architettura.

\end{itemize}
%slide 11
\end{document}