%%%%%%%%%%%%%%%%%%%%%%%%%%%%%%%%%%%%%%%%%
% Structured General Purpose Assignment
% LaTeX Template
%
% This template has been downloaded from:
% http://www.latextemplates.com
%
% Original author:
% Ted Pavlic (http://www.tedpavlic.com)
%
% Note:
% The \lipsum[#] commands throughout this template generate dummy text
% to fill the template out. These commands should all be removed when 
% writing assignment content.
%
%%%%%%%%%%%%%%%%%%%%%%%%%%%%%%%%%%%%%%%%%

\documentclass{article}

\usepackage{fancyhdr} % Required for custom headers
\usepackage{lastpage} % Required to determine the last page for the footer
\usepackage{extramarks} % Required for headers and footers
\usepackage{graphicx} % Required to insert images
\usepackage[utf8]{inputenc}

% Margins
\topmargin=-0.45in
\evensidemargin=0in
\oddsidemargin=0in
\textwidth=6.5in
\textheight=9.0in
\headsep=0.25in 

\linespread{1.1} % Line spacing



\setlength\parindent{0pt} % Removes all indentation from paragraphs

%----------------------------------------------------------------------------------------
%	DOCUMENT STRUCTURE COMMANDS
%	Skip this unless you know what you're doing
%----------------------------------------------------------------------------------------

% Header and footer for when a page split occurs within a problem environment
\newcommand{\enterProblemHeader}[1]{
\nobreak\extramarks{#1}{#1 continued on next page\ldots}\nobreak
\nobreak\extramarks{#1 (continued)}{#1 continued on next page\ldots}\nobreak
}

% Header and footer for when a page split occurs between problem environments
\newcommand{\exitProblemHeader}[1]{
\nobreak\extramarks{#1 (continued)}{#1 continued on next page\ldots}\nobreak
\nobreak\extramarks{#1}{}\nobreak
}

\setcounter{secnumdepth}{0} % Removes default section numbers
\newcounter{homeworkProblemCounter} % Creates a counter to keep track of the number of problems

%----------------------------------------------------------------------------------------
%	NAME AND CLASS SECTION
%----------------------------------------------------------------------------------------

\newcommand{\lessonNumber}[1]{Lezione\ \##1} % Assignment title
\newcommand{\lessonDate}[4]{#1,\ #2\ #3\ #4} % Due date
\newcommand{\lessonCourse}[1]{#1} % Course/class
\newcommand{\lessonTime}[1]{#1} % Class/lecture time
\newcommand{\lessonTeacher}[1]{#1} % Teacher/lecturer
\newcommand{\lessonAuthor}[1]{#1} % Your name
\begin{document}

\section{Documentazione (10)}

La documentazione deve essere una conseguenza dei processi organizzativi. Serve documentare nel modo meno intrusivo possibile. Fra le cose più importanti che fa la documentazione è fornire una misura sull'avanzamento del progetto (Piano di progetto), che fornisce dei \textbf{consuntivi} (stato fotografato) e \textbf{preventivi} (stime). I consuntivi sono finali se siamo alla fine del progetto, o parziali se siamo in corso. Documentare serve per dominare la complessità dei processi produttivi, attenuare la volatilità dei requisiti, facilitare il controllo di avanzamento.

Misureremo le cose sulle quali possiamo e vogliamo fissare degli obiettivi di miglioramento. Misurazione per obiettivi, "ad hoc". Il responsabile di progetto deve avere un ``\textit{cruscotto}'' con gli indicatori delle metriche che utilizziamo, e questi indicatori devono essere aggiornati. Vogliamo inoltre che il responsabile spenda il meno possibile per far questo devo documentare tutte le attività di pianificazione, gestione, sviluppo, verifica e validazione. Il piano ci dà degli obiettivi sui tempi e costi, le norme sono invece gli strumenti e le procedure che uso per rendere il piano fattibile. Il primo e più importante documento da realizzare (interno) sono le \textbf{norme di progetto}, che cresceranno nel tempo. Un altro documento importante è la specifica software, che è la descrizione ad alto livello del sistema.\\
Ogni architettura sw ha molte viste:

\begin{itemize}

	\item \textbf{Modello statico:} Identifica le componenti principali e procede per decomposizione grafica;
	\item \textbf{Modello dinamico:} Illustra la struttura a processi del sistema;
	\item \textbf{Modello delle interfacce:} Definisce le interfacce  del sistema;
	\item \textbf{Modello delle relazioni:} Identifica il flusso dei dati tra componenti distinti in relazione tra loro;
	\item \textbf{Modello di distribuzione:} Associazione tra nodi fisici e componenti logiche.

\end{itemize}

L'architettura va descritta in due documenti distinti:

\begin{itemize}

	\item \textbf{Specifica tecnica};
	\item \textbf{Definizione di prodotto:} ho bisogno di una \textit{milestone} che dica che questa è una buona architettura.

\end{itemize}
%slide 11
\end{document}