\input{../Templates/layout}
\input{../Templates/commands}
\begin{document}

\section{Diagrammi dei package(4)}

Sono usati spesso nella parte architetturale, ci permettono di tenere sotto controllo la complessità del sistema che si misura con le dipendenze. Un package non è altro che un raggruppamento di elementi UML. In effetti andremo a raggruppare solo classi. I rapporti tra package UML e package di linguaggio di programmazione è molto stretto, quasi 1:1. Il package si presenta come una grande cartella che ha un titolo e contiene al suo interno classi o altri package. Una classe appartiene ad un solo package. Il package individua un \textit{namespace}, ogni elemento deve avere un nome distinto all'interno dello spazio dei nomi. In UML per riferirsi ad un nome qualificato si usa la notazione dei :: .

\includegraphics[width=0.3\columnwidth]{img2} % Example image

L'interfaccia di un package è l'insieme delle classi che hanno visibilità pubblica di un package.\\
Principi di progettazione:

\begin{itemize}

	\item \textbf{Common Closure Principle}, classi dello stesso package condividono la stessa causa di cambiamento;
	\item \textbf{Common Reuse Principle}, classi dello stesso package dovrebbero sempre essere riusate insieme.

\end{itemize}

I package possono avere \textbf{dipendenze} tra loro:

\includegraphics[width=0.5\columnwidth]{img3} % Example image

Caratteristiche:

\begin{itemize}

	\item Tutte le dipendenze dovrebbero seguire la stessa direzione, a meno di isolamento voluto da sottostrutture;
	\item Evitare le dipendenze circolari;
	\item Relazioni di dipendenza non soltanto transitive, se modifico \textit{aaa} non necessariamente modifico \textit{ccc};
	\item Più dipendenze entranti, più il \textit{package} dovrebbe essere stabile.

\end{itemize}






\textbf{Esempio}:\\
\textit{Il cliente sfoglia il catalogo ed aggiunge i prodotti desiderati al carrello della spesa. Quando il cliente termina l’acquisto e deve pagare, lo stesso fornisce le informazioni sulla consegna dei prodotti e sulla carta di credito. Il sistema verifica l’autorizzazione al pagamento con carta di credito e conferma l’acquisto immediatamente e mediante una successiva mail.
}

\begin{center}

\includegraphics[width=0.5\columnwidth]{img4} % Example image

\end{center}

\begin{center}

\includegraphics[width=0.5\columnwidth]{img5} % Example image

\end{center}


\end{document}