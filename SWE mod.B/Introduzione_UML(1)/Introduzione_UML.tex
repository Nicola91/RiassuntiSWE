\input{../Templates/layout}
\input{../Templates/commands}
\begin{document}

\section{Introduzione a UML(1b)}

UML \fbox{\textbf{Def:} Unify Modeling Language} è un linguaggio per modellare concetti nel mondo ad oggetti. Una serie di notazioni puramente grafiche che hanno dei vincoli, c'è una sintassi e un meta-modello che serve a dare un significato alle notazioni grafiche (semantica). UML è puramente visuale è ha alcune caratteristiche:
\begin{itemize}

	\item \textbf{Astratto}, ognuno ha esperienze differenti su diversi linguaggi, quindi UML deve essere indipendente dal linguaggio di implementazione;
	\item \textbf{Flessibile}, si può parlare di qualsiasi cosa con UML;
	\item \textbf{Moderno}, progettato per sw scalabili, distribuiti e concorrenti. Scalabilità orizzontale e non verticale (in cui aggiungo risorse).

\end{itemize}
UML lo useremo in tutte le fasi di sviluppo sw. E' un standard ed è mantenuto dall'OMG, che incorpora molte aziende al suo interno. 
\end{document}