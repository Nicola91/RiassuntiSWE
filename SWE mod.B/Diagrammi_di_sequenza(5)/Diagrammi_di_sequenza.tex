\input{../Templates/layout}
\input{../Templates/commands}
\begin{document}

\section{Diagrammi di sequenza (5)}

Descrivono la collaborazione di un gruppo di oggetti che devono implementare collettivamente un comportamento.

\includegraphics[width=0.5\columnwidth]{img6} % Example image


Le iterazioni tra più partecipanti avvengono tramite \textbf{messaggi}. L'inizio di un messaggio si rappresenta con una freccia continua da un partecipante all'altro. E' possibile avere anche un messaggio che arriva dall'\textbf{esterno} del partecipante. L'evento esterno attiva il diagramma di sequenza e lo fa partire. Il ritorno di un messaggio è rappresentato da una freccia tratteggiata.\\
Esistono due possibili messaggi: 
\begin{itemize}
	\item \textbf{Asincroni}
	\item \textbf{Sincroni}
\end{itemize}

\includegraphics[width=0.5\columnwidth]{img5} % Example image


Un partecipante può creare un altro partecipante (normalmente tramite la \textit{new}), per questo si usa la \textbf{create}:

\includegraphics[width=0.5\columnwidth]{img7} % Example image


Di contro un oggetto può richiedere la distruzione di un altro oggetto. In questo caso si invia un altro messaggio e si mette una X sulla linea della vita del partecipante. Un oggetto può anche autodistruggersi.\\

Se ho bisogno di modellare la collaborazione uso i diagrammi di sequenza, se devo modellare algoritmi uso i diagrammi di attività.\\
Controllo \textbf{centralizzato} e \textbf{distribuito}. Bisogna cercare di delegare il più possibile agli oggetti le operazioni su se stessi (\textit{delegation pattern}. Nel contratto distribuito non abbiamo più un registro che \textit{schedula} tutte le attività, ma tutto parte da un singolo oggetto che invoca metodi su altri oggetti, manda e riceve messaggi.\\


\end{document}