\input{../Templates/layout}
\input{../Templates/commands}
\begin{document}
\section{Processi Software}
\textbf{Processi software:} attività coordinate, processi di ciclo di vita per far evolvere il sw da uno stato all'altro. Il sw è una macchina a stati che rappresentano il grado di maturazione del prodotto:

\begin{itemize}

	\item \textbf{Concezione}
	\item \textbf{Sviluppo}
	\item \textbf{Utilizzo}
	\item \textbf{Ritiro}

\end{itemize}

Nella fase di ritiro il sw cessa di esistere nel senso che non c'è più alcun tipo di supporto per quel prodotto. Le transizioni sono strettamente e formalmente regolate.	\\

\includegraphics[width=0.75\columnwidth]{img2} % Example image
\\

L'efficienza si vede dove vedo il consumo di risorse. L'efficacia si misura guardando i prodotti e vedendo se sono buoni o cattivi rispetto alla produzione. Un processo è un insieme di attività coordinate e coese (tutti hanno bisogno di tutti).\\
Parole chiave:
\begin{itemize}
	\item\textbf{Iterazione:} iterazione significa operare rivisitazioni o raffinamenti, può essere distruttivo (ripeto l'avanzamento) e rischio di non essere quantificabile.
	\item\textbf{Incremento:} Sono incrementale solo se aggiungo, mi avvicino in maniera monotona all'obbiettivo, non tollera errori o mancanze.
	\item\textbf{Prototipo:}servono per provare e scegliere soluzioni, possono essere o usa e getta o fornire stati di incremento(\textbf{baseline}).
	\item\textbf{Riuso} può essere di due tipi:
	\begin{itemize}
	
		\item occasionale, a basso consto ma a basso impatto.
		\item sistematico, a maggior costo ma a maggior impatto.
	\end{itemize}
	\item \textbf{Manutenzione:} va gestita con il controllo di versione che va ben documentato.
\end{itemize}

\end{document}
